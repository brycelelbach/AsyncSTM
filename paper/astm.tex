\documentclass[conference]{IEEEtran}

\usepackage{xcolor}
\definecolor{darkred}{rgb}{0.5,0,0}
\definecolor{darkgreen}{rgb}{0,0.5,0}
\definecolor{darkblue}{rgb}{0,0,0.5}

\usepackage{graphicx}
\usepackage{amsmath}
\usepackage{amssymb}
\usepackage{xspace}
\usepackage{upgreek}
\usepackage{multirow}

\usepackage{caption}
\usepackage[ampersand]{easylist}

\usepackage{fixme}
\fxusetheme{color}
\fxsetup{
    status=draft,
    author=,
    layout=inline, % also try footnote or pdfnote
    theme=color
}

\definecolor{fxnote}{rgb}{0.8000,0.0000,0.0000}

\usepackage[bordercolor=white,backgroundcolor=gray!30,linecolor=black,colorinlistoftodos]{todonotes}
\newcommand{\rework}[1]{%
    \todo[color=yellow,inline]{{#1}}%
}
\newcommand{\I}[1]{\textit{#1}}
\newcommand{\B}[1]{\textbf{#1}}
\newcommand{\T}[1]{\texttt{#1}}
\newcommand{\F}[1]{\B{\textcolor{red}{FIXME: #1}}}

\usepackage{listings}
\lstloadlanguages{C++,Pascal}

% Settings for the lstlistings environment
\lstset{
language=C++,                       % choose the language of the code
basicstyle=\footnotesize\ttfamily,  % the size of the fonts that are used for the
                                    % code
numbers=none,                       % where to put the line-numbers
numberstyle=\tiny,                  % the size of the fonts that are used for the
                                    % line-numbers
stepnumber=1,                       % the step between two line-numbers. If it's
                                    % 1 each line will be numbered
numbersep=5pt,                      % how far the line-numbers are from the code
%backgroundcolor=\color{gray},      % choose the background color. You must add
                                    % \usepackage{color}
showspaces=false,                   % show spaces adding particular underscores
showstringspaces=false,             % underline spaces within strings
showtabs=false,                     % show tabs within strings adding particular
                                    % underscores
keywordstyle=\bfseries\color{darkblue},  % color of the keywords
commentstyle=\color{darkgreen},     % color of the comments
stringstyle=\color{darkred},        % color of strings
captionpos=b,                       % sets the caption-position to top
tabsize=2,                          % sets default tabsize to 2 spaces
frame=tb,                           % adds a frame around the code
breaklines=true,                    % sets automatic line breaking
breakatwhitespace=false,            % sets if automatic breaks should only happen
                                    % at whitespace
escapechar=\%,                      % toggles between regular LaTeX and listing
belowskip=0.3cm,                    % vspace after listing
morecomment=[s][\bfseries\color{darkblue}]{struct}{\ },
morecomment=[s][\bfseries\color{darkblue}]{class}{\ },
morecomment=[s][\bfseries\color{darkblue}]{public:}{\ },
morecomment=[s][\bfseries\color{darkblue}]{public}{\ },
morecomment=[s][\bfseries\color{darkblue}]{protected:}{\ },
morecomment=[s][\bfseries\color{darkblue}]{private:}{\ },
morecomment=[s][\bfseries\color{black}]{operator+}{\ },
xleftmargin=0.1cm,
%xrightmargin=0.1cm,
}

\newcommand{\mus}{$\upmu$s\xspace}

\newcommand{\code}[1]{\texttt{{{#1}}}}

\begin{document}

\title{ASTM Paper}

\author{
\IEEEauthorblockN{Bryce Adelstein-Lelbach\IEEEauthorrefmark{1}, Steve Brandt\IEEEauthorrefmark{1}}
\IEEEauthorblockA{\small \IEEEauthorrefmark{1}Center for Computation and Technology, Louisiana State University}
\IEEEauthorblockA{\scriptsize \tt blelbach@cct.lsu.edu, sbrandt@cct.lsu.edu}
}

\maketitle


\begin{abstract}
\fxnote{INSERT ABSTRACT HERE}
\end{abstract}



\IEEEpeerreviewmaketitle
% no \IEEEPARstart
%This demo file is intended to serve as a ``starter file'' for IEEE conference papers produced under \LaTeX\ using IEEEtran.cls version 1.7 and later.
% You must have at least 2 lines in the paragraph with the drop letter


\section{Introduction}

% Introduction (1 page)
%   * What is the problem?
%   * Explain the basic idea of STM
%   * Talk about applications of STM
%       * Concurrent data structures for "system programming"
%           * OS/kernel/runtime systems
%           * Ex: video game frameworks, RCU trees in Linux
%   * The problem: STM is limited to side effect free code
%   * Literature review: what people have tried
%   * Overview of our approach
%

\subsection{HPX}

HPX is a general purpose parallel runtime system for applications of any scale.
It exposes a homogeneous programming model which unifies the execution of remote
and local operations. The runtime system has been developed for conventional
architectures. Currently supported are SMP nodes, large Non Uniform Memory Access
(NUMA) machines and heterogeneous systems such as clusters equipped with Xeon Phi
accelarators. Strict adherence to Standard C++~\cite{cxx11_standard} and the
utilization of the Boost C++ Libraries~\cite{boostcpplibraries} makes HPX both
portable and highly optimized.
The source code is published under the Boost Software
License~\fxnote{cite} making it accessible to everyone as Open Source Software.
It is modular, feature-complete and designed for
best possible performance. HPX's design focuses on overcoming conventional
limitations such as (implicit and explicit) global barriers, poor latency hiding,
static-only resource allocation, and lack of support for medium- to fine-grain
parallelism.

\section{Technical Approach}

Our system consists of three primary components. Transaction objects
represents a single transaction. They are responsible for orchestrating reads and
writes that occur in a transaction attempt, and they manage commit
attempts. A transaction object owns the storage for transaction-local
variables. Data that is used in transactions is represented by
transaction variables wrappers. These wrappers proxy an underlying type,
providing the functionality necessary for usage inside of a transaction.
Finally, transaction futures represent continuations launched within
transactions.

\subsection{Transaction Objects}

A transaction object is an entity that represents a single transaction
block. In our system, a transaction block is formed in native C++ using a
do-while loop and a transaction object:

$
transaction t
do {
    // Transaction block.
} while (!t.commit_transaction());
$

The transaction object has two functions:

\begin{itemize}
\item Provides transaction-local storage: during transaction attempts, the
shared variables which are accessed by the transaction need to be recorded. The
initial value of read-variables needs to be stored to check for transaction
failure during the commit, and updates to write-variables need to be buffered.
Additionally, any continuations launched within the transaction need to be
buffered so that they can be launched on a successful commit.
\item Implement the four basic operations which occur in transactions:
\textbf{read}, \textbf{write}, \textbf{async}, and
\textbf{commit}.
\end{itemize}

The transaction object uses four data structures to implement transaction-local
storage: a map containing the local state of transaction variables and three
structures which store the information needed to process a commit attempt (one
for reads, one for writes and one for async operations). 

The local state of all variables accessed during the course of a transaction
are stored in an ordered map. This local state is indexed by the memory address
of the transaction wrapper object which is being accessed. Upon insertion into
the map, the transaction wrapper is deep-copied. Every time a variable is
externally read, the value which is read is addded to a read list. The elements
of this list have the same key-value structure of the variable map. Whenever an
internal write operation occurs, an entry is added to a write set to keep track
of which entries in the variable map will need to be written externally upon
commit.

The read operation instructions requests access to a local copy a transaction
variable. If the variable is not present in the transaction's local state, a
thread-safe read of the actual variable value will occur. This value will be
added to the variable map and the read list. If the variable has been written
prior to the read in the transaction, no external read is performed. Instead,
the buffered value from the write will be read from the local variable map
and the map entry will be added to the read list.

The write operation is mostly symmetric to the read operation. A local write to
a variable that has not been previously accessed will create a new entry in the
variable map, but no external read will be made. A write operation will add the
variable's memory reference to the write set.

The async operation is the novel aspect of our system. This operation will
buffer a continuation, which will be launched asynchronously upon the
successful commit of the transaction. The async operation takes two
arguments; a bound function to be invoked, and a reference to the future
object which the operation will be appended to. The future object can be
thought of as a queue; when a transaction commits, callbacks are simply
appended to their futures using .then(). When the action associated with the
future becomes ready, the continuation will begin immediately.

TODO: Description of operations

TODO: Concurrency gurantees/order of locking

\subsection{Shared Variable Wrappers}

The shared variable wrapper is a wrapper object that represents an object of an
arbitrary wrapped type which can be used in a transaction. The shared variable
wrapper has the following functions in our system:

\begin{itemize}
\item Safely type-erases objects that are used in transactions. The wrapper is
part of a type erasure system which enables a type-agnostic implementation of
the transaction object.
\item Provides an interface to the user for accessing and updating variables
inside and outside of transactions.
\item Enforces the invarants regarding variable access in our system.
\end{itemize}

\subsection{Transaction Futures}

TODO

\subsection{Variable Queues}

\section{Applications/Results}

\section{Future Work/Conclusions}

% use section* for acknowledgement
\section*{Acknowledgment}

%\bibliographystyle{IEEEtran}
%\bibliography{astm}

% that's all folks
\end{document}



\end{document}
