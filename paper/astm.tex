\documentclass[conference]{IEEEtran}

\usepackage{xcolor}
\definecolor{darkred}{rgb}{0.5,0,0}
\definecolor{darkgreen}{rgb}{0,0.5,0}
\definecolor{darkblue}{rgb}{0,0,0.5}

\usepackage{graphicx}
\usepackage{amsmath}
\usepackage{amssymb}
\usepackage{xspace}
\usepackage{upgreek}
\usepackage{multirow}

\usepackage{caption}
\usepackage[ampersand]{easylist}

\usepackage{fixme}
\fxusetheme{color}
\fxsetup{
    status=draft,
    author=,
    layout=inline, % also try footnote or pdfnote
    theme=color
}

\definecolor{fxnote}{rgb}{0.8000,0.0000,0.0000}

\usepackage[bordercolor=white,backgroundcolor=gray!30,linecolor=black,colorinlistoftodos]{todonotes}
\newcommand{\rework}[1]{%
    \todo[color=yellow,inline]{{#1}}%
}
\newcommand{\I}[1]{\textit{#1}}
\newcommand{\B}[1]{\textbf{#1}}
\newcommand{\T}[1]{\texttt{#1}}
\newcommand{\F}[1]{\B{\textcolor{red}{FIXME: #1}}}

\usepackage{listings}
\lstloadlanguages{C++,Pascal}

% Settings for the lstlistings environment
\lstset{
language=C++,                       % choose the language of the code
basicstyle=\footnotesize\ttfamily,  % the size of the fonts that are used for the
                                    % code
numbers=none,                       % where to put the line-numbers
numberstyle=\tiny,                  % the size of the fonts that are used for the
                                    % line-numbers
stepnumber=1,                       % the step between two line-numbers. If it's
                                    % 1 each line will be numbered
numbersep=5pt,                      % how far the line-numbers are from the code
%backgroundcolor=\color{gray},      % choose the background color. You must add
                                    % \usepackage{color}
showspaces=false,                   % show spaces adding particular underscores
showstringspaces=false,             % underline spaces within strings
showtabs=false,                     % show tabs within strings adding particular
                                    % underscores
keywordstyle=\bfseries\color{darkblue},  % color of the keywords
commentstyle=\color{darkgreen},     % color of the comments
stringstyle=\color{darkred},        % color of strings
captionpos=b,                       % sets the caption-position to top
tabsize=2,                          % sets default tabsize to 2 spaces
frame=tb,                           % adds a frame around the code
breaklines=true,                    % sets automatic line breaking
breakatwhitespace=false,            % sets if automatic breaks should only happen
                                    % at whitespace
escapechar=\%,                      % toggles between regular LaTeX and listing
belowskip=0.3cm,                    % vspace after listing
morecomment=[s][\bfseries\color{darkblue}]{struct}{\ },
morecomment=[s][\bfseries\color{darkblue}]{class}{\ },
morecomment=[s][\bfseries\color{darkblue}]{public:}{\ },
morecomment=[s][\bfseries\color{darkblue}]{public}{\ },
morecomment=[s][\bfseries\color{darkblue}]{protected:}{\ },
morecomment=[s][\bfseries\color{darkblue}]{private:}{\ },
morecomment=[s][\bfseries\color{black}]{operator+}{\ },
xleftmargin=0.1cm,
%xrightmargin=0.1cm,
}

\newcommand{\mus}{$\upmu$s\xspace}

\newcommand{\code}[1]{\texttt{{{#1}}}}

\begin{document}

\title{ASTM Paper}

\author{
\IEEEauthorblockN{Bryce Adelstein-Lelbach\IEEEauthorrefmark{1}, Steve Brandt\IEEEauthorrefmark{1}}
\IEEEauthorblockA{\small \IEEEauthorrefmark{1}Center for Computation and Technology, Louisiana State University}
\IEEEauthorblockA{\scriptsize \tt blelbach@cct.lsu.edu, sbrandt@cct.lsu.edu}
}

\maketitle


\begin{abstract}
\fxnote{INSERT ABSTRACT HERE}
\end{abstract}



\IEEEpeerreviewmaketitle
% no \IEEEPARstart
%This demo file is intended to serve as a ``starter file'' for IEEE conference papers produced under \LaTeX\ using IEEEtran.cls version 1.7 and later.
% You must have at least 2 lines in the paragraph with the drop letter


\section{Introduction}

% Introduction (1 page)
%   * What is the problem?
%   * Explain the basic idea of STM
%   * Talk about applications of STM
%       * Concurrent data structures for "system programming"
%           * OS/kernel/runtime systems
%           * Ex: video game frameworks, RCU trees in Linux
%   * The problem: STM is limited to side effect free code
%   * Literature review: what people have tried
%   * Overview of our approach
%       * HPX

\section{Technical Approach}

Our STM system consists of three primary components. $transaction$ objects
represents a single transaction. It is responsible for orchestrating reads and
writes that ooccur in a single transaction attempt, and it manages commit
attempts. A $transaction$ owns the storage for transaction-local
storage. Variables that are used in transactions are represented by
$shared\_var$ wrappers.  $shared\_var$s wrap an underlying type,
providing the functionality necessary for usage inside of a transaction.
Finally, $transaction_future$s represent continuations launched within
transactions.

\subsection{Transaction Objects}

A $transaction$ object is an entity that represents the a single transaction
block. In our system, a transaction block is formed in native C++ using a
do-while loop and a transaction object ~\ref{?}.

$
transaction t
do {
    // Transaction block.
} while (!t.commit_transaction());
$

The transaction object has two functions:

\begin{itemize}
\item Provides transaction-local storage: during transaction attempts, the
shared variables which are accessed by the transaction need to be recorded. The
initial value of read-variables needs to be stored to check for transaction
failure during the commit, and updates to write-variables need to be buffered.
Additionally, any continuations launched within the transaction need to be
buffered so that they can be launched on a successful commit.
\item Implement the four basic operations which occur in transactions:
\textbf{read}, \textbf{write}, \textbf{async}, \textbf{abort} and
\textbf{commit}. TODO
\end{itemize}

TODO: Description of operations

TODO: Concurrency gurantees/order of locking

\subsection{Shared Variable Wrappers}

The shared variable wrapper is a wrapper object that represents an object of an
arbitrary wrapped type which can be used in a transaction. The shared variable
wrapper has the following functions in our system:

\begin{itemize}
\item Safely type-erases objects that are used in transactions. The wrapper is
part of a type erasure system which enables a type-agnostic implementation of
the transaction object.
\item Provides an interface to the user for accessing and updating variables
inside and outside of transactions.
\item Enforces the invarants regarding variable access in our system.
\end{itemize}

\subsection{Transaction Futures}

TODO

\section{Applications/Results}

\section{Future Work/Conclusions}

% use section* for acknowledgement
\section*{Acknowledgment}

\bibliographystyle{IEEEtran}
\bibliography{IEEEabrv,astm}

% that's all folks
\end{document}



\end{document}
